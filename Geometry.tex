\section{Geometry}
\Algorithm{Primitive Geometry}
{Templates for Points, Lines, and Circles}
{$\mathcal{O}(1)$ for each type of operation}
{cpp}{source/template/PrimitiveGeometry.cpp}

\Algorithm{Manhattan MST}
{MST for Mahattan distance, by choosing $\mathcal{O}$(n) candidates for edges}
{$\mathcal{O}(n \ log \ n)$}
{cpp}{source/template/ManhattanMST.cpp}

\Algorithm{Convex Hull}
{Convex Hull: The smallest polygon that has every point of the set inside it. Convex Hull is also a set of convex combinations between every points:
$CH(S) = \{ p^* \ | \ \exists \alpha_i \ \text{such that } \ \sum_{i=1}^n\alpha_i=1,\sum_{i=1}^na_ip_i\},$}
{$\mathcal{O}(n \ log \ n)$}
{cpp}{source/template/ConvexHull.cpp}

\Algorithm{Line Hull Intersection}
{Split the convex hull into two halves, and check if the line intersects with the convex hull}
{O(log n)}
{cpp}{source/template/linehull.cpp}

\subsection{Geometry Formula}
\textbf{Euler Formula}: For any planar graph with v vertices, e edges, and f faces, we have v - e + f = 2

\textbf{Pick Theorem}: $A = I + 1/2 B - 1$, A: Area, I: Internal lattic points, B: boarder lattic points.

\textbf{In-circle determinant test}: a Point D is on the circumcircle of point A, B, C if A, B, C, D are coplanar when mapping as followed: $(x, y) \to (x, y, x^2 + y^2)$ (Code available in Delaunay Triangulation)

\textbf{Geometrical Inversion:} When dealing with circles, one can use inverse technique to turn:
\begin{itemize}
    \item a line that doesn't pass through the center to a circle through the center.
    \item a circle goes through the center to a line that doesn't go through the center.
    \item a circle that doesn't go through the center to another circle.
\end{itemize}

\textbf{Descartes Theorem}: For every 4 mutually tangent circle, the following holds: $(\sum_{i=1}^4k_i)^2 = 2 \sum_{i=1}^4k_i^2$

\textbf{Heron}: Area of a triangle of side a, b, c: $\sqrt{p(p-a)(p-b)(p-c)}$, where p is half the perimeter

\textbf{Stewart Theorem}: Let a, b, c be the lengths of the sides of a triangle. Let d be the length of a cevian to the side of length a. If the cevian divides the side of length a into two segments of length m and n, with m adjacent to c and n adjacent to b, then $b^2m+c^2n=a(d^2+mn)$

\textbf{Location of Important Points}:
\begin{itemize}
    \item centroid: $G = (A + B + C) / 3$
    \item Incenter: $I = \frac{(aA + bB + cC)}{(a + b + c)}$ 
    \item Circumcenter \\ 
    $\frac{(x_B^2 + y_B^2 - x_A^2 - y_A^2)(y_C - y_A) - (x_C^2 + y_C^2 - x_A^2 - y_A^2)(y_B - y_A)}{D}$ \\
    $\frac{(x_B - x_A)(x_C^2 + y_C^2 - x_A^2 - y_A^2) - (x_C - x_A)(x_B^2 + y_B^2 - x_A^2 - y_A^2)}{D}$ \\
    $$ D = 2 \left[ (x_B - x_A)(y_C - y_A) - (x_C - x_A)(y_B - y_A) \right] $$
    \item Orthocenter: $O(x_O, y_O) (\text{circumcenter)}$:$$x_H = x_A + x_B + x_C - 2x_O$$$$y_H = y_A + y_B + y_C - 2y_O
$$
\end{itemize}

\Algorithm{AREA}
{Area of Union of Rectangles}
{$\mathcal{O}(n \ log \ n)$}
{cpp}{source/template/Area.cpp}

\subsection{Delaunay Triangulation}
\textbf{USAGE}: Build the Delaunay triangulation configuration / Voronoi Diagram:
\begin{itemize}
\item The Euclidean MST is a subset of Delaunay Triangulation Edges.
\item From Delaunay Triangulation, one can construct the Voronoi Diagram: The planar subdivision such that every point in a region is closest to that one point.
\end{itemize}
{O(n log n) with Divide-And-Conquer Algorithm}
\inputminted[]{cpp}{source/template/Delaunay.cpp}


\Algorithm{Half-plane Intersection}
{Find the half-plane intersection of many lines: Find the solution of a so-called linear inequalities}
{O(n log n)}
{cpp}{source/template/halfplane.cpp}
